%-----------------------------------------------
%This is a basic template of the graduate thesis
%for student who are at Dependability Engineering Laboratory.
%
% created by Hideharu Kojima
%-----------------------------------------------

\documentclass[11pt,a4j]{jreport}
\usepackage{thesis_ja}%use thesis.sty for using layout for the thesis
\usepackage{graphicx}%
\usepackage{Here}
  
\renewcommand{\baselinestretch}{1.3}%

\thesisauthor{小島英春}%
\affiliation{大阪大学 基礎工学部 情報科学科}%
\thesistitle{あれこれ}%
\supervisor{土屋 達弘}%


\abstcontents{
ここに内容梗概,ここに内容梗概,ここに内容梗概,ここに内容梗概,ここに内容梗概,ここに内容梗概
ここに内容梗概,ここに内容梗概,ここに内容梗概,ここに内容梗概,ここに内容梗概,ここに内容梗概
ここに内容梗概,ここに内容梗概,ここに内容梗概,ここに内容梗概,ここに内容梗概,ここに内容梗概
ここに内容梗概,ここに内容梗概,ここに内容梗概,ここに内容梗概,ここに内容梗概,ここに内容梗概
ここに内容梗概,ここに内容梗概,ここに内容梗概,ここに内容梗概,ここに内容梗概,ここに内容梗概
ここに内容梗概,ここに内容梗概,ここに内容梗概,ここに内容梗概,ここに内容梗概,ここに内容梗概
ここに内容梗概,ここに内容梗概,ここに内容梗概,ここに内容梗概,ここに内容梗概,ここに内容梗概
ここに内容梗概,ここに内容梗概,ここに内容梗概,ここに内容梗概,ここに内容梗概,ここに内容梗概
ここに内容梗概,ここに内容梗概,ここに内容梗概,ここに内容梗概,ここに内容梗概,ここに内容梗概
}
\keywords{本研究に関するキーワードを記載,コンマで区切る}


%\maketitle

\begin{document}
%\makepreambles generate a title page, abstract pages and index pages.
\makepreambles

\chapter{はじめに}
\section{なんやかんや}


\subsection{どうとかこうとか}
図は以下のように
\begin{verbatim}
\begin{figure}[H]
\begin{center}
\includegraphics[scale=0.45]{images/image.eps}
\caption{図}
\label{fig:modeling_topology}
\end{center}
\end{figure}
\end{verbatim}
Plain text.


\subsection{Another subtitle}
表\ref{tbl:table1}は以下のように
\begin{verbatim}
\begin{table}[H]ここのオプションはHhtb,Hはhere.styを使うことで利用可能
\begin{center}
\caption{表}
\label{tbl:table1}
\begin{tabular}{|c|l|}\hline
(a)&どうこう\\\hline
(b)&あれそれ\\\hline
(c)&どれそれ\\\hline
(d)&これ\\\hline
\end{tabular}
\end{center}
\end{table}
\end{verbatim}
\begin{table}[H]
\begin{center}
\caption{表の例}
\label{tbl:table1}
\begin{tabular}{|c|l|}\hline
(a)&どうこう\cite{AODV}\\\hline
(b)&あれそれ\cite{DSR}\\\hline
(c)&どれそれ\cite{OLSR}\\\hline
(d)&これ\cite{mycom}\cite{ISSTA1994}\\\hline
\end{tabular}
\end{center}
\end{table}
More plain text\cite{ICTE2007}.



\acknowledgement%
ここに謝辞かきます
%\begin{bibliography}



%\end{bibliography}

\bibliographystyle{sieicej}
\bibliography{bibsample.bib}
\end{document}
